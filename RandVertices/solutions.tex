\def\inputGnumericTable{}
\let\negmedspace\undefined
\let\negthickspace\undefined
\documentclass[journal,12pt,onecolumn]{IEEEtran}
\usepackage{cite}
\usepackage{amsmath,amssymb,amsfonts,amsthm}
\usepackage{algorithmic}
\usepackage{graphicx}
\usepackage{textcomp}
\usepackage{xcolor}
\usepackage{txfonts}
\usepackage{listings}
\usepackage{enumitem}
\usepackage{mathtools}
\usepackage{gensymb}
\usepackage[breaklinks=true]{hyperref}
\usepackage{tkz-euclide} % loads  TikZ and tkz-base
\usepackage{listings}
\usepackage{gvv}

  \usepackage[latin1]{inputenc}
  \usepackage{color}                                            %%
  \usepackage{array}                                            %%
  \usepackage{longtable}                                        %%
  \usepackage{calc}                                             %%
  \usepackage{multirow}                                         %%
  \usepackage{hhline}                                           %%
  \usepackage{ifthen}                                           %%
  %optionally (for landscape tables embedded in another document): %%
  \usepackage{lscape}    

\newtheorem{theorem}{Theorem}[section]
\newtheorem{problem}{Problem}
\newtheorem{proposition}{Proposition}[section]
\newtheorem{lemma}{Lemma}[section]
\newtheorem{corollary}[theorem]{Corollary}
\newtheorem{example}{Example}[section]
\newtheorem{definition}[problem]{Definition}
\newcommand{\BEQA}{\begin{eqnarray}}
\newcommand{\EEQA}{\end{eqnarray}}
\newcommand{\define}{\stackrel{\triangle}{=}}
\theoremstyle{remark}
\newtheorem{rem}{Remark}

%\bibliographystyle{ieeetr}
\setlength{\parindent}{0pt}
\begin{document}
\bibliographystyle{IEEEtran}

\vspace{3cm}
\title{
%	\logo{
EE23010 Assignment
%	}
}
\author{ Sayyam Palrecha$^{*}$ EE22BTECH11047% <-this % stops a space
	\thanks{}	
}

% make the title area
\maketitle
%\tableofcontents
\bigskip
\renewcommand{\thefigure}{\theenumi}
\renewcommand{\thetable}{\theenumi}

Consider a triangle with vertices:
\begin{align}
\vec{A} = \myvec{1\\-1},
\vec{B} = \myvec{-4\\6},
\vec{C} = \myvec{-3\\-5}
\end{align}
\counterwithin{figure}{section}
\counterwithin{table}{section}
\begin{table}[!ht]
	\section{vectors}
	\centering
	\input{./tables/Q1.1.tex}
	\caption{Vectors}
	\label{table:vectors}
\end{table}

\begin{figure}
\includegraphics[width=\columnwidth]{./figs/Q1.1.3.png}
\caption{Triangle $ABC$ ($\triangle ABC$)}
\label{fig:vectors}
\end{figure}

\begin{table}[!ht]
	\section{median}
	\centering
	\input{./tables/Q1.2.tex}
	\caption{Median}
	\label{table:median}
\end{table}

\begin{figure}
\includegraphics[width=\columnwidth]{./figs/Q1.2.2.png}
\caption{Medians $AD$, $BE$ and $CF$ with centroid $\vec{G}$ of $\triangle ABC$}
\label{fig:median}
\end{figure}

\begin{table}[!ht]
	\section{altitude}
	\centering
	\input{./tables/Q1.3.tex}
	\caption{Altitude}
	\label{table:Altitude}
\end{table}

\begin{figure}
\includegraphics[width=\columnwidth]{./figs/Q1.3.4.png}
\caption{Altitudes $AD_1$, $BE_1$ and $CF_1$ with orthocentre $\vec{H}$ of $\triangle ABC$}
\label{fig:altitude}
\end{figure}

\begin{table}[!ht]
	\section{perpendicular bisector}
	\centering
	\input{./tables/Q1.4.tex}
	\caption{Perpendicular Bisector}
	\label{table:Perpendicular Bisector}
\end{table}

\begin{figure}
\includegraphics[width=\columnwidth]{./figs/Q1.4.1.png}
\caption{Perpendicular bisectors $OD$, $OE$ and $OF$ with circumcentre $\vec{O}$ of $\triangle ABC$}
\label{fig:perpendicular bisector}
\end{figure}

\begin{table}[!ht]
	\section{angle bisector}
	\centering
	\input{./tables/Q1.5.tex}
	\caption{Angle Bisector}
	\label{table:Angle Bisector}
\end{table}

\begin{figure}
\includegraphics[width=\columnwidth]{./figs/Q1.5.1.png}
\caption{Angle bisectors of $\angle A$, $\angle B$ and $\angle C$ with incentre $\vec{I}$ of $\triangle ABC$}
\label{fig:angle bisector}
\end{figure}

\end{document}
